\chapter{Background}\label{C:back}

To investigate the feasibility and potential pathways, I started by looking into general applied machine learning practices and academic papers about already existing systems like the one I was pursuing. Not to my surprise, I found that this process of automating essay scoring has been a subject of study for a while, specifically within education and business. An early note-able pursuit within this field dates back to the 1960s where Ellis Batten Page's PEG, Project Essay Scorer, began to gain public and scientific momentum and acceptance [2]. The systems that are products of this area are called AES's - Automatic Essay Scorers. A good AES can deliver scores that are relatively similar to human ratings. For example, the Educational Testing Service's e-rater system, also known as "e-rater," can produce accurate results over 90$\%$ percent of the time [3]. 

\section{Machine Learning}

AES systems have incorporated a sub-field of Artificial intelligence, Machine Learning, from early as 2013 [4]. Predictive Machine Learning is the science of making computers intelligent enough to make their own decisions. It has been around for a long time, but it has become powerful enough that it can now be used on various problems in recent years. Predictive Machine learning is generally used for tasks where it is beneficial for a computer to understand a relationship between an input and an output for a specific problem [5]. This is often used to use this relationship understanding to synthesize an output when only given an input. In our case, this relationship is between a text, and it is the corresponding score. We want to refine this relationship between the input and output to be given an unmarked essay and produce a score. 

\section{Models}

This relationship is called a model and is defined as a description of a system defined with mathematical concepts and language [6]. The model of an AES is a critical component of the overall solution. There are several types within the field of Machine Learning and AES. They can be divided into two categories. The first category consists of relatively simple models that employ statistical calculations such as Regression or Classification. Regression can be used to predict continuous values, for example, a score of "5.0" or "3.1" [7]. Classification can be used to identify, separate, or categorise values into discrete values, such as "true" or "false" [8]. The second category consists of relatively more complex models. These models work by simulating a network of interconnected nodes that work similarly to neurons in the human brain. To recognize and learn correlations within the data. They can then be used to predict continuous or discrete data [9].

\section{Feature Extraction}

Each model requires its input to be in a process-able format and descriptive of the problem before it can mathematically evaluate it. This process of optimizing the input to a more interpretable format is called Feature Extraction [10]]. When using a Neural Network model, this step can be skipped entirely [11]. However, when working with Regression and Classification models, this step must be performed manually using Manual Feature Engineering. This requires identifying and programmatically developing Features from an essay that are thought to indicate the skills being assessed [12. In our case, a feature can be thought of as a characteristic, property, or attribute of an essay. For example, the number of words in the text, the number of grammatical errors, or the average length of the words used. This process is done one feature at a time, requires domain knowledge, and can be time-consuming and error-prone. 

\section{Model Choice}

\paragraph{Prediction Type}
The output alone influenced my decision to deviate from a classification model. As, fundamentally, classification is about predicting a categorical variable. However, this type of output variable could be used to address the first approach mentioned in the introduction by, for example, classifying a student into either a "doing well" or "not doing well" group for a marking-criteria category. Neural and Regression models can do better by producing continuous outputs or predictions that align better with this project's preferred approach as they would produce a continuous output for a marking-criteria category.  

\paragraph{Nature of Relationship}
The nature of the relationship that is being investigated is also a crucial consideration. In some applications, neural network models can perform better than linear regression models when there are non-linearities involved. This is because neural networks, when appropriately configured, can understand highly complex and convoluted non-linear relationships. [13]. Non-linearity is a term used in statistics to describe a situation where there is no straight-line or direct relationship between two variables. In a non-linear relationship, changes in the output do not directly change to changes in any of the inputs [14]. In this study, I expect to observe linear relationships between the features and the output due to the nature of the problem of this project. I would expect to see that as the number of grammatical errors increases, the computer-generated score will decrease. This influenced my decision to lean toward a Regression model as they can perform well with linear relationships[x]. Additionally, they can do so with less configuration and, therefore, domain knowledge [x].

\paragraph{The Data}
The data available also plays a considerable role in narrowing down model options. As its size and relevance can influence a models' ability to synthesize the relationship in question. This is especially true for Neural Network models due to their better generalised ability to learn. As for where more data is available, better results can be observed from Neural network models [x]. However, in cases where this is not true, a Regression Model can be a better choice. For the duration of this project, nine datasets were used. Of these datasets, only one directly represented the problem and was not available for the entire duration of the project either. For these reasons, I saw a Regression Model being a better choice.

\paragraph{Feature Extraction}
Feature Extraction is another critical consideration when choosing a model. When considering a Neural model, the possibility of skipping the process of Manual Feature Engineering entirely can sound like an immediate win. However, despite Manual Feature Engineering introducing another component into the mix, it can provide much value. Especially when dealing with smaller datasets. Lilja observed a performance decrease in their study with Neural Models. And suggested that human insight when working with linguistic features could be crucial when dealing with small sample sizes [15]. Additionally, I'm optimistic about Manual Feature Engineering as I believe most features will not require extensive domain knowledge and are also relatively subjective. For these reasons, from the perspective of the available data, I saw value in Manual Feature Engineering. 

\paragraph{Explain-ability}
Lastly, explain-ability is an essential factor in this consideration as it aligns with the high-level goal of this project. Explainability is the concept that a machine learning model and its output can be explained in a way that "makes sense" to a human being at an acceptable level. When considering Manual Feature Engineering alone, there are already gains in explainability as the Feature Extraction step would be exposed and become the researchers' responsibility. This would mean that this step can be more easily explained to stakeholders. In addition, to provide insight, such as; what the features are and for which calculations they are being used for. Additionally, even without Manual Feature Engineering, Regression models are more accessible to explain than Neural Network models [x]. For these reasons, from the explaibabity, I saw a Regression with Manual Feature Engineering Model being a better choice.

\paragraph{}As discussed in the previous sections, Manual Feature Engineering paired with a Regression Model is the most optimal starting point for this project.
